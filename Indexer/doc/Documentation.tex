\documentclass[11pt]{article}
%Gummi|065|=)
\usepackage{graphicx,url}
\usepackage[utf8]{inputenc}
\usepackage[brazil]{babel}
\title{\textbf{Trabalho Prático II - Indexador}}
\author{Thiago Vieira de Alcantara Silva\\2012075627}
\date{}
\begin{document}

\maketitle

\section{Introdução}
Neste trabalho, desenvolvi um indexador e um processador de consultas simples. Dado um conjunto de páginas web, o indexador as analisa, retira todos os termos presentes em seus conteúdos e cria um arquivo invertido, onde cada linha representa um termo e contém pares indicando em qual documento podemos encontrar o termo e quantas vezes ele aparece no documento. O processador de consultas foi desenvolvido seguindo o modelo booleano, assim as consultas são realizadas somente verificando se os termos da consulta estão presentes nos documentos; o ranking dos documentos é dado pelo número que termos da consulta que eles contém.

\section{Decisões de Projeto}
\subsection{Estrutura do Projeto}
Tanto o indexador como o processador de consultas foram desenvolvidos em C++.
A pasta principal do projeto é dividida em oito pastas diferentes:
\begin{itemize}
\item bin: Pasta onde os binários do indexador se localizam.
\item doc: Pasta para guardar a descrição e documentação do projeto.
\item documents: Pasta que uso para guardar sets de páginas web, uso não obrigatório.
\item gumbo-parser: Pasta presente no indexador, onde a biblioteca gumbo é armazenada.
\item output: Pasta onde os arquivos finais se localizam, dentre eles estão o index, urls, vocabulary$\_$i e vocabulary$\_$l, onde guardo, respectivamente, o arquivo invertido, a lista de urls, o vocabulário ordenado pelos índices e o ordenado lexicograficamente
\item src: Pasta onde guardo o código fonte do indexador.
\item tst: Pasta onde ficam os testes realizados para aprender as interfaces das bibliotecas usadas e para avaliar a corretude e eficiência de partes do código.
\item .tmp: Pasta usada para guardar arquivos temporários necessários para o indexador.
\end{itemize}

\subsection{Bibliotecas, Dependências e Execução}
Os pacotes necessários para executar o indexador são gumbo-parser e libicu. O gumbo-parser  é uma biblioteca desenvolvida pela Google ( \url{http://github.com/google/gumbo-parser} ), usada para fazer o parsing das páginas web e a libicu é utilizada para retirar caracteres especiais dos documentos.\\
Um shell script, install.sh, foi desenvolvido para checar se o sistema operacional contém os pacotes e os instala se necessário.
Para facilitar a compilação do indexador foi criado um snippet usando o utilitário Makefile, logo para compilar o código basta executar um make que o objeto main aparecerá na pasta principal.\\
No arquivo DOCFILENAMES está localizada a lista dos nomes dos arquivos onde estão as páginas web, não é necessário passar este arquivo para o executável do indexador. Ao iniciar, o indexador lê os nomes dos arquivos de DOCFILENAMES, os armazena e começa o processo de indexação.

\subsection{Método de Ordenação Externa}
Após a geração das triplas, um algoritmo de ordenação entra em ação. Para ordenar o arquivos das triplas foi usado o heap-sort multiway, assim ordenamos chunks do arquivo de triplas e os salvamos em arquivos separados, logo após, uma intercalação desses arquivos é executada gerando o arquivo final ordenado.\\


\section{Análise de Complexidade}

%TODO: 
% Analisar complexidade de memoria
% Fazer analise assintotica em relacao ao tempo
% Fazer analise da quantidade de escritas feitos em disco.
% Falar da demora da geracao de triplas
% 


\section{Resultados Experimentais}
\subsection{Ordenação Externa}
%TODO: 
% Mostrar a velocidade 
% 
% 
% 
% 


\section{Conclusão}
%TODO: 
% 
% 
% 
% 
% 


\end{document}
