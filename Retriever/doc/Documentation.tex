\documentclass[11pt]{article}
%Gummi|065|=)
\usepackage{graphicx,url}
\usepackage[utf8]{inputenc}
\usepackage[brazil]{babel}
\usepackage{graphicx}
\title{\textbf{Trabalho Prático III - Buscador}}
\author{Thiago Vieira de Alcantara Silva\\2012075627}
\date{}
\begin{document}

\maketitle

\section{Introdução}
Neste trabalho, desenvolvi um processador de consultas usando um modelo vetorial. Dado um conjunto de dados e uma consulta, o processador retira os termos da consulta formando um vetor, então calculamos a similaridade da consulta com cada documento. Os documentos mais similares são considerados como resposta.\\
Para a execução das consultas desenvolvemos uma página html onde o usuário pode inserir suas consultas e obter os documentos mais similares a ela.


\section{Decisões de Projeto}
\subsection{Estrutura do Projeto}
O processador de consultas foi desenvolvido em C++.
A pasta principal do projeto é dividida em seis pastas diferentes:
\begin{enumerate}
\item base: Nesta pasta temos todas as classes que são usadas pelos buscadores implementados no TPII ou TPIII.
\item boolean: Nesta pasta temos o buscador implementado para o segundo trabalho prático.
\item doc: Na pasta doc guardamos a descrição e a documentação do trabalho.
\item search: Em search podemos encontrar as páginas html que podem ser usadas para executar consultas.
\item util: Na pasta util temos scripts utilitários que servem para executar tarefas específicas.
\item vectormodel: Nesta pasta temos o buscador implementado para o terceiro trabalho prático. Este buscador recupera 20 documentos e os ordena de acordo com a similaridade com a consulta executada.
\end{enumerate}

\subsection{Arquivos Auxiliares, Dependências e Execução}
O processador em vectormodel depende somente das classes da base.\\
Para compilar o processador de consultas basta executar o utilitário Make que se encontra dentro da pasta vectormodel. Existe uma opção de execução disponível, podemos informar o nome do arquivo em que queremos salvar a resposta do processador com a option -o. Assim se executarmos ./main -o [Nome do arquivo] o resultado da consulta será salvo em um arquivo.\\
Na pasta vectormodel temos um arquivo chamado CONFIG, nele guardamos o nome do arquivo que contém o índice, o nome do arquivo que contém a lista de urls e o nome do arquivo que contém o vocabulário da base de dados indexada. Ao executar o binário do processador de consultas, a primeira coisa que ele faz é ler o arquivo CONFIG e carregar os arquivos descritos nele.


\section {Ranking}
A escolha e rankeamento dos arquivos que devem ser retornados após uma consulta é feita usando um cálculo de similaridade entre o vetor da consulta e os documentos da base de dados indexada. O cálculo de similaridade feito neste trabalho foi o cosseno. Um vetor de um documento ou consulta pode ser enxergado como um vetor n-dimensional onde cada dimensão representa uma palavra do vocabulário, a magnitude do valor de cada dimensão do vetor é calculada com base na frequência que cada termo aparece no documento.
Assim o documento mais relevante de acordo com o processador será o que tiver o maior valor de cosseno com a consulta.


\section{Análise de Complexidade}
Buscas são feitas percorrendo o índice invertido e guardando o peso de cada documento em memória principal, assim para cada 1 milhão de páginas na base de dados, gastaremos pouco mais de 1 Mbytes de dados por consulta executada.

Em relação ao tempo, como para cada tripla do índice atualizamos o valor do vetor dos documentos, a complexidade assintótica de uma busca é O($I$ * log $n$) onde $I$ é o número de triplas do índice invertido e $n$ é o número de documentos da base de dados.



\end{document}
